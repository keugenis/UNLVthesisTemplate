\chapter[: Where Do I Get \LaTeX?]{Where Do I Get \LaTeX?}

\LaTeX\ is a freeware application.  It was originally designed by Donald Knuth at Stanford, and is being maintained by the (mostly European) academic community.  There are versions that run on PC, Mac, and Linnux systems, among others.  Though the compilers are platform dependent, the source files are not. 

\section{Getting \LaTeX Up and Running}

A great place to start is the \LaTeX-project website site about obtaining \LaTeX\. It is here: \url{https://latex-project.org/ftp.html}. The way you proceed is usually going to involve installing a version of \LaTeX\ and then installing an editor.

\section{Getting your \TeX\ distribution}

You will need to install your \TeX\ distribution before you try to install an editor. For most beginning users, you'll want to install using the default options. The original PC version of \LaTeX\ is MikTeX.  It can be obtained from \url{www.miktex.org}. You can also try proTeXt, which is a \TeX\ distribution for Windows that is based on MikTeX and adds a few extra features. You can find this here: \url{http://www.tug.org/protext}.

The \TeX\ distribution for Mac OS is called MacTeX. It can be downloaded here: \url{http://www.tug.org/mactex/}. If you're using Linux, you probably already have a TeX system installed. If you don't, you should install TeX Live from here: \url{http://www.tug.org/texlive}.

\section{Choosing and installing an editor}

 Then, you'll probably want to choose \LaTeX\ editor. Although MikTex and MacTex both come with front ends, there are heaps of very good open source (free!) editors from which to choose. Dr.\ Gill prefers TeXstudio, which is free from \url{http://texstudio.sourceforge.net}. This editor supports Windows, Mac OS, and Linux. TeXworks (\url{https://www.tug.org/texworks}) and TeXmaker (\url{http://www.xm1math.net/texmaker}) are also very good, and they work on all of the major operating systems. Dr.\ Baragar prefers Winedt (\url{http://www.winedt.com}), which works on Windows. 
 
 Sometimes, you might want to make some edits on the fly. For this, you might want to use an online, web-based editor. For this, my favorite is ShareLaTeX. It has a free version and a subscription version. However, if you share ShareLaTeX with others, you can unlock the premium features for free! So, if you get ShareLaTeX from this link, you'll help me unlock free Dropbox integration! \url{https://www.sharelatex.com?r=8c76f322&rm=d&rs=b}. Other good online editors are Overleaf \url{https://www.overleaf.com} and Authorea \url{https://www.authorea.com}, both of which also have free versions and paid subscription-based versions.
 
 Of course, things are always changing in the world of \LaTeX\ . Dr.\ Gill compiled this list as of 2016, but there may be even better options available to you now. Be sure to search for something like ``Best \LaTeX\ Editors 20XX'' (where the 20XX is the current year) to see if you can find something new.
 
\section{Getting started}

It's probably best to extract these files into a new folder for your thesis or dissertation. You should keep a copy of the originals, though, in case you accidentally change something that sends the entire thing down a spiral of warnings, errors, and failed rendering. If your dissertation is temporarily called ``Project X,'' you would extract all of these into a folder of that name.

From there, you'll edit the content in the main files. You might want to change the names of the chapter files. If you do, be sure to change the names in the $\backslash include\{SampleChapterX\}$ entry in the UNLVthesisTemplate.tex file. You're probably going to need more than three chapters, so you'll need to create more and then add $\backslash include\{SampleChapterX\}$ entries in the UNLVthesisTemplate.tex file. 

You will also probably (hopefully!) need a bibliography. Dr.\ Gill has formatted this template to use \texttt{natbib}, which is my favorite. You might want to do this differently, depending upon which formatting style you need to use for your works cited list. Natbib is very flexible (especially for author-date formats). You can find more information on customizing your bibliography using natbib by consulting the Reference Sheet on CTAN's natbib page: \url{http://mirrors.ctan.org/macros/latex/contrib/natbib/natnotes.pdf}.

\section{Resources}

Probably the best place to start is on the starter page for the website \url{www.ctan.org/starter}. This includes links to {\it The (Not so) short introduction to
\LaTeX2$\epsilon$}, by Tobias Oiteker, et al.\ It is a 110 page free
document \citep{oetiker2015notsoshort}. You'll also want to bookmark the \LaTeX\ Wikibook: \url{https://en.wikibooks.org/wiki/LaTeX}. The nice thing about this resource is that it is updated regularly with new information. There are also countless tutorials and videos on the internet. Google is your friend!

There are several books out there that might also be helpful. Some of the old favorites might work, and they might help you avoid having to use a lot of the newer packages. The downside is that these books are out of date about these new packages, so you might miss out on a very simple way to do what you need to do. With that caveat, here are some of the classics courtesy of Dr.\ Baragar: One perennial favorite is {\it The \LaTeX\ Companion} by \citet{goossens1994latex}, but this is probably more for an expert. {\it A document preparation system \LaTeX} by \citet{lamport1986document} is more suited to the beginner. There is also {\it A guide to \LaTeX}, by \citet{kopka2004guide}, but Dr.\ Baragar thinks this, too, is for the expert. 

Because of the rapidly evolving nature of open source programming languages, you may prefer to use the online guides and discussions at places like the \TeX\ section of StackExchange (\url{http://tex.stackexchange.com/questions/ask}). Search first and familiarize yourself with the protocol before you ask a question; the folks over there get irritated if you ask a duplicate question or fail to provide a minimal working example (MWE)!
